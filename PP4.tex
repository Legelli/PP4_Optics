% !TEX TS-program = XeLaTeX
%!TEX encoding = UTF-8 Unicode
%==================================================
%      PREAMBOLO e DICHIARAZIONI INIZIALI
%==================================================
\documentclass[10pt,oneside,a4paper]{article}

\usepackage[utf8]{inputenc} 
\usepackage[italian]{babel}
\usepackage[T1]{fontenc}
\usepackage{siunitx} %Inserisce automaticamente i dati con le unità  di misura correttamente formattate del SI (utilizzo: \SI{0.82}{m^2}, in generale \SI{misura con il punto decimale}{unità  di misura})
\sisetup{output-decimal-marker = {.}, separate-uncertainty = true, input-uncertainty-signs = \pm, detect-weight=true, detect-family=true} %per usare SI con il punto decimale
\usepackage{listings} %Per citare codice informatico formattandolo correttamente
\usepackage{amsmath,amsthm,verbatim,amssymb,amsfonts,amscd,graphicx,mathtools}
\usepackage[makeroom]{cancel}
\newcommand{\abs}[1]{\left\lvert\,#1\,\right\rvert}
\usepackage{geometry}
\usepackage{epigraph}
\usepackage{booktabs}	%tabelle migliorate
\usepackage{tablefootnote}	%note a piè di pagina in tabella
\usepackage{threeparttable} %tabella con note a piè di tabella
\usepackage{caption}	%descrizione per figure
\usepackage{dblfnote}
\usepackage{supertabular}
\usepackage{longtable}
\captionsetup{tableposition=top,figureposition=bottom,font=small} %setup descrizione
\usepackage{float}
\usepackage{esvect} %vettori
\usepackage{longtable} %tabelle lunghe
\usepackage[dvipsnames]{xcolor}
\definecolor{sepia}{HTML}{80002A}
\usepackage[colorlinks=true, citecolor=black, linkcolor=sepia, urlcolor=black]{hyperref}
\usepackage{mathrsfs}
\usepackage{circuitikz}
\tikzset{
  font={\fontsize{7pt}{12}\selectfont}}
\ctikzset{bipoles/resistor/height=0.2}
\ctikzset{bipoles/resistor/width=0.4}
\ctikzset{bipoles/diode/height=0.3}
\ctikzset{bipoles/diode/width=0.3}
\ctikzset{tripoles/american nand port/height=0.7}
\ctikzset{tripoles/american nand port/width=0.8}
\usepackage{enumitem} %Liste senza spazi verticali
\setlist{noitemsep}
\usepackage{amsmath}
\usepackage{hyperref}
%\usepackage{pst-optexp} %Diagrammi ottici
\usepackage{physics} %Ambienti utili
\usepackage{upgreek} %Per avere lettere greche non corsive, ex. \upbeta


\interfootnotelinepenalty=10000


\usepackage{multicol}
\newenvironment{Figure}
  {\par\medskip\noindent\minipage{\linewidth}}
  {\endminipage\par\medskip}

%\newcommand{\var}{\operatorname{var}}
%\newcommand{\cov}{\operatorname{cov}}


\usepackage{listings} %Per inserire codice
\lstdefinestyle{CStyle}{
    backgroundcolor=\color{backgroundColour},   
    commentstyle=\color{mGreen},
    keywordstyle=\color{magenta},
    numberstyle=\tiny\color{mGray},
    stringstyle=\color{mPurple},
    basicstyle=\footnotesize\ttfamily,
    breakatwhitespace=false,         
    breaklines=true,                 
    captionpos=b,                    
    keepspaces=true,                 
    numbers=left,                    
    numbersep=5pt,                  
    showspaces=false,                
    showstringspaces=false,
    showtabs=false,                  
    tabsize=2,
    language=C
}

\definecolor{color1}{RGB}{90,0,0} % Color of the article title and sections
\definecolor{color2}{RGB}{0,20,50} % Color of the boxes behind the abstract and headings
\definecolor{mGreen}{rgb}{0,0.6,0}
\definecolor{mGray}{rgb}{0.5,0.5,0.5}
\definecolor{mPurple}{rgb}{0.58,0,0.82}
\definecolor{backgroundColour}{rgb}{0.95,0.95,0.92}


%==================================================
%                  PRIMA PAGINA
%==================================================

\title{\textsc{\textbf{Esperienza 4}: Interferometro di Fabry-Perot}}
\author{\small{G. Galbato Muscio} \and \small{F. Ghimenti} \and \small{L. Gravina} \and \small{L. Graziotto}}
\date{21 Maggio 2019}

\begin{document}
	\begin{figure}
		\centering
		\includegraphics[scale=0.5, trim={2.8cm 8.9cm 0 9cm}, clip]{logo.png}
	\end{figure}
	\maketitle
	\begin{center} 
		\fbox{{\fontsize{12pt}{8mm}\textsc{Gruppo D1-1}}} \\
	\end{center}
\hrule
\vfill
\renewcommand{\abstractname}{Abstract}
\begin{abstract}
Si studia la funzione di trasmissione di un interferometro di Fabry-Perot e, con un fit alla funzione di Airy, se ne ricava la \emph{finesse} di riflettività. Variando la riflettività degli specchi dell'interferometro, si fornisce una seconda stima della \emph{finesse}. Infine, variando la distanza tra gli specchi, si stima la lunghezza d'onda del laser He-Ne impiegato.
\end{abstract}
\vfill
\tableofcontents %Indice
\newpage


\pagebreak


\begin{multicols}{2}
%==================================================
%             APPARATO STRUMENTALE
%==================================================
\section{Apparato strumentale}

Si utilizza un laser He-Ne di lunghezza d'onda, dichiarata dal costruttore, $\lambda = \SI{632.8}{nm}$, montato su tavolo ottico\footnote{Si confronterà dunque il risultato sperimentale ottenuto in seguito con questo valore.}. 

In serie al laser è posta un'iride, allo scopo di evitare l'ingresso nel laser dei fasci di ritorno, che ne perturberebbero il comportamento. Due specchi orientati a \SI{45}{\degree} portano il fascio ad incidere sull'interferometro di Fabry-Perot. Il secondo specchio dell'interferometro è posto su una slitta regolabile con una vite micrometrica; inoltre, sulla slitta è posto un cristallo piezoelettrico ai cui capi è applicato un segnale a rampa, al fine di passare da un picco all'altro della funzione di trasmissione. 

Quindi, a distanza\footnote{L'incertezza associata è pari al doppio della risoluzione del metro a nastro, in quanto si ha un errore dovuto sia al posizionamento di un capo dello strumento, sia al posizionamento dell'altro.} $L = \SI{209 \pm 2}{mm}$ rispetto al secondo specchio dell'interferometro, è posto il fotodiodo, montato su una slitta micrometrica di portata \SI{15}{mm} e risoluzione \SI{0.010}{mm}, che può essere traslato per misurare l'intensità luminosa delle frange di interferenza.

La configurazione utilizzata è illustrata in Figura~\ref{fig:diagram}.

\begin{Figure}
	\begin{center}
	\hbox{\hspace{-0.8cm}
	\includegraphics[width=1.1\linewidth]{diagram.png}}
	\captionof{figure}{Configurazione utilizzata.}
	\label{fig:diagram}
	\end{center}
\end{Figure}

Il segnale in uscita dal fotodiodo è inviato al \texttt{CH 2} dell'oscilloscopio \texttt{Tektronik TDS2012C}. Le misure di intensità luminosa vengono riportate come differenza di potenziale misurata ai capi del fotodiodo, pertanto è da intendere la presenza di un fattore di proporzionalità non noto. Inoltre, si regola con un filtro attenuatore l'intensità della luce emessa dal laser in modo da restare all'interno della regione di linearità del fotodiodo, ossia al di sotto di \SI{10}{V}. L'incertezza associata alle misure mediante i cursori è quella fornita dal manuale\footnote{\url{http://pdf1.alldatasheet.com/datasheet-pdf/view/554089/ETC2/TDS2012C.html}} dell'oscilloscopio, ossia il $3\%$.

Poiché il fotodiodo permette l'ingresso della luce attraverso un foro di diametro circa \SI{200}{\micro m}, si compiranno spostamenti della slitta micrometrica di almeno \SI{100}{\micro m}, e pertanto l'incertezza associata alla posizione del fotodiodo sarà di \SI{100}{\micro m}.

%==================================================
%             FIT DELLA FUNZIONE DI AIRY
%==================================================
\section{Misura della funzione di trasmissione e fit alla funzione di Airy}


%==================================================
%             MISURA DELLA FINESSE
%==================================================
\section{Misura della finesse al variare delle combinazioni di specchi}
Si vuole misurare la \emph{finesse} dell'interferometro, definita come 
\begin{equation}\label{eq:finesse}
	\mathcal{F} = \pi \frac{\sqrt{R} }{1-R}
\end{equation}
essendo $R$ la media geometrica delle riflettività dei due specchi utilizzati per costruire l'interferometro. Assumendo che la dilatazione della ceramica sia lineare rispetto alla tensione $V$ applicata, si deduce che la variazione di fase $\Delta$ tra le onde piane uscenti dall'interferometro, proporzionale alla variazione di distanza $D$ tra i due specchi dell'interferometro, è anch'essa lineare in $V$, cioè si assume che
\[
	\Delta = b V.
\]
Per stimare il coefficiente di proporzionalità $b$ si prendono le misure $V$ e $\tilde{V}$ di tensione corrispondenti a due massimi consecutivi che siano sulla stessa rampa dell'onda triangolare, uguagliando le intensità corrispondenti ai due diversi ordini si trova che
\begin{equation}\label{eq:b}
	b=\frac{2\pi}{\tilde{V}-V}.
\end{equation}
Inoltre, imponendo che l'intensità sia metà di quella massima, e nell'approssimazione in cui $\Delta V b/ 2<< 1$, essendo $\Delta V$ la semilarghezza del picco della funzione di Airy corrispondente ad intensità dimezzata, si trova
\begin{equation}\label{eq:F_approx}
	\mathcal{F} \approx \frac{\tilde{V} - V}{2 \Delta V}.
\end{equation}
Si effettuano le misure di tensione per diverse combinazioni di specchi, le misure sono riportate in Tabella \ref{tab:parte_due}. Si riporta in Figura \ref{fig:parte_due_retta} $\mathcal{F}^{-2}$, con $\mathcal{F}$ stimata da (\ref{eq:F_approx}), in funzione di $\mathcal{F}_\mathrm{teo}^{-2}$, dove $\mathcal{F}_\mathrm{teo}$ è la finesse teorica data da (\ref{eq:finesse}), in Figura \ref{fig:parte_due_F} si riporta invece l'andamento della finesse sperimentale in funzione della media geometrica delle riflettività utilizzate.
\begin{Figure}
	\begin{center}
	\includegraphics[width=\linewidth]{parte_due.pdf}
	\captionof{figure}{Andamento del quadrato del reciproco della finesse sperimentale in funzione del quadrato del reciproco della finesse ideale}
	\label{fig:parte_due_retta}
	\end{center}
\end{Figure}
\begin{Figure}
	\begin{center}
	\includegraphics[width=\linewidth]{parte_due_F.pdf}
	\captionof{figure}{Andamento della finesse sperimentale in funzione della media geometrica delle riflettività dei due specchi dell'interferometro, si riporta la finesse ideale per confronto}
	\label{fig:parte_due_F}
	\end{center}
\end{Figure}
Come è evidente dalla Figura \ref{fig:parte_due_F} , le misure si discostano molto dall'andamento ideale. Ciò è inoltre confermato dalla Figura \ref{fig:parte_due_retta}, nel quale ci si aspettava un andamento lineare con coefficiente angolare unitario e intercetta positiva. Si conclude che questo risultato sia dovuto ad un errore sistematico nell'allineamento del sistema: l'allineamento è stato infatti realizzato sistematicamente al fine di massimizzare la visibilità delle frange, ma senza preoccuparsi di allineare il centro della figura di interferenza con il fotodiodo, una misura conclusiva fatta allineando correttamente il sistema ha infatti restituito il valore teorico atteso.


%==================================================
%             MISURA DI LAMBDA
%==================================================
\section{Misura della lunghezza del Fabry-Perot e stima della lunghezza d'onda del laser}
Si verifica che per tutta la durata di questa presa dati il voltaggio del fotodiodo rimane compreso tra \SI{0}{} e \SI{0.6}{V}, dunque nella regione di linearità. Si rammenta che la distanza tra il fotodiodo e il secondo specchio del Fabry-Perot è $L=\SI{209 \pm 2}{mm}$; si interpone dunque tra il secondo specchio metallico e l'interferometro un foglio di plastica trasparente opaco quale mezzo diffusore, al fine di ottenere un'onda analoga a quella generata da una sorgente puntiforme. Oltre l'interferometro si osserva dunque il pattern di massimi e minimi circolari detto \emph{anelli di Airy}.

Si verifica che la posizione assoluta del secondo specchio dell'interferometro, rispetto allo zero della vite micrometrica che non necessariamente corrisponde alla distanza nulla tra gli specchi, è $d_1 = \SI{13.37 \pm 0.01}{mm}$.

Traslando la slitta su cui è posto il fotodiodo si misurano le posizioni di due massimi consecutivi e se ne stima l'angolo rispetto al centro del pattern come $\theta = (x-x_0) / L$. Si ha la relazione tra ordine del massimo e angolo 
\begin{equation}\label{eq:diff}
\frac{2d^f}{\lambda} \cos{\theta_m} = m
\end{equation}
sviluppando il coseno al secondo ordine si ottiene, dalla misura di due angoli,
\[
m \simeq \frac{2}{\theta_{m-1}^2 - \theta_{m}^2}
\]
e conoscendo il valore della lunghezza d'onda $\lambda$ (come dichiarato dal costruttore) si ottiene una stima della lunghezza del Fabry-Perot
\[
d^f \simeq \frac{m \lambda}{2},
\]
valida per $m \gg 1$.

Variando quindi la posizione assoluta del secondo specchio del Fabry-Perot, e portandolo a $d_2 = \SI{10.75 \pm 0.01}{mm}$, si ripete il procedimento precedente e si ricava l'ordine del massimo $m_2$. Quindi, si può stimare la lunghezza d'onda del laser da \[
\lambda^{\text{exp}} = \frac{2(d_1 - d_2)}{m_2 - m_1}.
\]

Si riportano in Tabella~\ref{tab:stimaLambda} i dati raccolti. La stima della lunghezza d'onda risulta essere
\[
\lambda^{\text{exp}} = \SI{ 181 \pm 46}{nm},
\]

E le corrispondenti lunghezze $d_i^f$ del Fabry-Perot

$$d_1^f = \SI{2.8 \pm 0.9}{mm}$$
$$d_2^f = \SI{0.21 \pm 0.07}{mm}$$

Il termine dominante nel calcolo delle incertezze risulta essere quello relativo all'ordine di diffrazione. Osserviamo che il valore di $lambda$ determinato sperimentalmente non è compatibile con quello dichiarato dal costruttore. Nemmeno le lunghezze del Fabry Perot sono coerenti con quanto osservato durante l'esperimento, dove l'ordine di grandezza della separazione degli specchi è di circa \SI{1}{cm}. Consideriamo i dati raccolti quando lo specchio mobile dell'interferometro si trova a una distsanza assoluta $d_2$. Secondo la relazione \ref{eq:diff} all'aumentare di $m$ la distanza relativa tra due massimi consecutivi dovrebbe ridursi per via della funzione coseno. Questo non accade nei dati sperimentali raccolti, dove $x_1-x_0 < x_2-x_1$. Ciò porta a pensare a errori di lettura nella presa dati, in particolare di $x_2$, a causa di un non corretto allineamento del fascio.
Utilizzando invece la lunghezza d'onda fornita dal costruttore $\lambda = \SI{632.8}{nm}$\footnote{\url{https://docs.google.com/viewer?a=v&pid=sites&srcid=ZGVmYXVsdGRvbWFpbnxvdHRpY2FlbGFib3JhdG9yaW9lZGVscmV8Z3g6MWE0NWExZDNlOTVhMzQ2ZA}} e gli ordini ricavati $m_1$ e $m_2$ si ottiene rispettivamente
$$d_1^f = \SI{1 \pm 0.2}{cm} $$
$$d_2^f = \SI{0.73 \pm 0.03}{mm}$$

Osserviamo che $L_1-L_2$ non è compatibile con $d_1 - d_2$. Ciò rafforza l'ipotesi di errori nella stima dell'ordine di diffrazione per la seconda configurazione.

\begin{center}
\begin{table*}
\captionof{table}{Misure per la stima di $\lambda$}
\label{tab:stimaLambda}
\begin{tabular}{c|c|c|c|c|c}
 & $x_0$ [\SI{}{mm}] & $x_1$ [\SI{}{mm}] & $x_2$ [\SI{}{mm}] & $\theta_1$  & $\theta_2$ \\
 & $\pm 0.01$ & $\pm 0.1$ & $\pm 0.1$ & & \\
\hline 
$d_1 = \SI{13.37 \pm 0.01}{mm}$ & 10.70  & 12.7 & 11.8 &    \SI{9.6 \pm 0.7 e-3}{} & \SI{5.3 \pm 1.0 e-3}{} \\
$d_2 = \SI{10.75 \pm 0.01}{mm}$ & 14.80   & 13.1 & 8.4  & \SI{8.1 \pm 0.7 e-3}{} & \SI{31.0 \pm 0.7 e-3}{} \\
\hline
\end{tabular}
\begin{tabular}{c|c|c}
& m & $d^f$ \\
\hline
$d_1 = \SI{13.37 \pm 0.01}{mm}$ & \SI{31 \pm 7 e3}{} & \SI{2.8 \pm 0.9}{mm}\\
$d_2 = \SI{10.75 \pm 0.01}{mm}$ & \SI{2.3 \pm 0.1 e3}{}  & \SI{0.21 \pm 0.07}{mm}\\
\hline
\end{tabular}
\end{table*}
\end{center}









\end{multicols}


\newpage
\section{Appendice}



%ESEMPIO DI FIGURA
%\begin{Figure}
%	\begin{center}
%	\includegraphics[width=\linewidth]{comune.png}
%	\captionof{figure}{Istantanea dell'oscilloscopio per l'amplificatore differenziale, misura di $A_c$}
%	\label{fig:Ac_differenziale}
%	\end{center}
%\end{Figure}


%ESEMPIO DI TABELLA
%\begin{center}
%\captionof{table}{Misure per la stima di $A_c$}
%\label{tab:Ac_differenziale}
%\begin{tabular}{c|c|c|c}
%$f$ [\SI{}{Hz}] & $V_i$ [\SI{}{V}] & $v_o$ [\SI{}{mV}] & $A_c = v_o / V_i$ \\
%\hline
%      149.5 &        3.90 &         11.3 & 2.90e-03 \\
%      222.0 &        3.90 &         11.5 & 2.95e-03 \\
%      281.0 &        3.90 &         11.8 & 3.03e-03 \\
%      359.0 &        3.90 &         11.8 & 3.03e-03 \\
%      461.0 &        3.90 &         12.2 & 3.13e-03 \\
%\hline
%\end{tabular}
%\end{center}


\end{document}
